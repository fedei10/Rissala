\chapter{Introduction} 

%remove everything here ==============================>
\section{Introduction Generale : 
}
la dominance du contenu videographique absurde et sans valeur ajouté a été l'environnement ideal pour la naissance d'un nouveau format de contenu qui est
\section{les newsletters }
un court article ecrit par des createurs pour partager leurs connaissances et leurs experience. 
Dans ce cadre s'inscrit notre projet Rissala, qui est une plateforme de newsletters dédiée aux créateurs de contenu, leur permettant de partager leurs connaissances et leurs expériences par e-mail ou directement via leur profil sur Rissala.

\section{Qu'est-ce qu'une newsletter ?}
Les newsletters regroupent des articles offrant aux créateurs de contenu un moyen direct et personnel de rester en contact avec leur public. Elles les tiennent informés des dernières actualités, des nouveaux projets et leur permettent de partager des contenus exclusifs réservés aux abonnés, sans les contraintes d'une plateforme.
\section{Comment ça marche ?}
En s'abonnant à la newsletter d'un créateur de contenu, notre plateforme met à jour une base de données contenant leurs e-mails et envoie directement les articles créés par e-mail, tout en sauvegardant ces articles dans leur profil.
\section{Pourquoi une newsletter ?}
Une newsletter n'est pas soumise à des restrictions ni contrôlée par une plateforme, offrant ainsi aux créateurs de contenu la liberté de partager du contenu avec une audience ciblée et certainement intéressée.
\section{Le futur des newsletters :}
Face à la domination du contenu vidéographique et des contenus courts sur les réseaux sociaux, nous avons ressenti un besoin de relations directes et authentiques avec nos créateurs. L'absence de lectures, puisque de moins en moins de personnes lisent des livres, souligne l'importance des newsletters comme moyen d'adaptation. Ces articles, généralement de 2 à 8 pages, offrent une alternative engageante et informative dans un paysage numérique saturé 